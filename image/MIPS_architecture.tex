\documentclass[12pt]{standalone}
\usepackage{circuitikz}

\usetikzlibrary{shapes.geometric}

\ctikzset{multipoles/flipflop/clock wedge size=0.4}

\tikzset{register/.style={flipflop,
    flipflop def={t2=$\,$, t5=$\,$, cu=1},
    transform shape,
    scale=.5
}}

\tikzset{two io register/.style={flipflop,
    flipflop def={t1=$\,$, t3=$\,$, t4=$\,$, t6=$\,$, cu=1},
    transform shape,
    scale=.5
}}

\tikzset{two input mux/.style={muxdemux,
    muxdemux def={Lh=4, Rh=2, NL=2, NT=1, NB=0, NR=1},
    scale=.4
}}

\tikzset{four input mux/.style={muxdemux,
    muxdemux def={Lh=6, Rh=4, NL=4, NT=1, NB=0, NR=1},
    scale=.4
}}

\ctikzset{multipoles/dipchip/width=2.5,
          multipoles/dipchip/pin spacing = 0.75}

\begin{document}

\begin{tikzpicture}
    
    %COMPONENTS---------------

    %Program Counter
    \node[register] (PC) at (0,0) {PC};
    
    %PC Mux
    \node[two input mux] (PC_mux) at ($ (PC) + (2,-.225) $) {};
    \node[right, font=\tiny] at (PC_mux.lpin 1) {0};
    \node[right, font=\tiny] at (PC_mux.lpin 2) {1};

    %Instr/Data Memory
    \node[dipchip, num pins=6, hide numbers, no topmark, align=center, external pins width=0] (memory) at ($ (PC_mux) + (3,0) $) {\scriptsize\textbf{Instr/Data} \\ \scriptsize\textbf{Memory}};
    \draw (memory.bpin 1) ++(0,0.2) -- ++(0.2,-0.2) node[right, font=\tiny] {CLK} -- ++(-0.2,-0.2);
    \node[right, font=\tiny] at (memory.bpin 2) {A};
    \node[right, font=\tiny] at (memory.bpin 3) {WD};
    \node[left, font=\tiny] at (memory.bpin 5) {RD};
    \node[below, font=\tiny] at (memory.n) {WE};

    %Instruction register
    \node[register] (IR) at ($ (memory.bpin 5) + (2,0) $) {IR};

    %Register File
    \node[dipchip, num pins=10, hide numbers, no topmark, align=center, external pins width=0] (regfile) at ($ (IR) + (6.5,-1) $) {\scriptsize\textbf{Register} \\ \scriptsize\textbf{File}};
    \draw (regfile.bpin 1) ++(0,0.2) -- ++(0.2,-0.2) node[right, font=\tiny] {CLK} -- ++(-0.2,-0.2);
    \node[right, font=\tiny] at (regfile.bpin 2) {A1};
    \node[right, font=\tiny] at (regfile.bpin 3) {A2};
    \node[right, font=\tiny] at (regfile.bpin 4) {A3};
    \node[right, font=\tiny] at (regfile.bpin 5) {WD3};
    \node[left, font=\tiny] at (regfile.bpin 9) {RD2};
    \node[left, font=\tiny] at (regfile.bpin 10) {RD1};
    \node[below, font=\tiny] at (regfile.n) {WE3};

    %RegDst Mux
    \node[two input mux] (RegDst_mux) at ($ (regfile.bpin 4) - (2,0) $) {};
    \node[right, font=\tiny] at (RegDst_mux.lpin 1) {0};
    \node[right, font=\tiny] at (RegDst_mux.lpin 2) {1};
    
    %MemToReg Mux
    \node[two input mux] (MemToReg_mux) at ($ (regfile.bpin 5) - (1,0) $) {};
    \node[right, font=\tiny] at (MemToReg_mux.lpin 1) {0};
    \node[right, font=\tiny] at (MemToReg_mux.lpin 2) {1};
    
    %Data register
    \node[register] (Data) at (IR |- MemToReg_mux.lpin 2) {Data};

    %Operand register
    \node[two io register] (Operand) at ($ (regfile.bpin 9) !0.5! (regfile.bpin 10) + (2,0) $) {Operand};
    \node[left, font=\tiny] at ($ (Operand.pin 4) - (.05,0) $) {B};
    \node[left, font=\tiny] at ($ (Operand.pin 6) - (.05,0) $) {A};

    %SrcA mux
    \node[two input mux] (SrcA_mux) at ($ (Operand.pin 6) + (1,.225) $) {};
    \node[right, font=\tiny] at (SrcA_mux.lpin 1) {0};
    \node[right, font=\tiny] at (SrcA_mux.lpin 2) {1};

    %SrcB mux
    \node[four input mux] (SrcB_mux) at ($ (Operand.pin 4) + (2,-.505) $) {};
    \node[right, font=\tiny] at (SrcB_mux.lpin 1) {00};
    \node[right, font=\tiny] at (SrcB_mux.lpin 2) {01};
    \node[right, font=\tiny] at (SrcB_mux.lpin 3) {10};
    \node[right, font=\tiny] at (SrcB_mux.lpin 4) {11};

    %ALU
    \node[ALU, scale=.8, font=\tiny] (ALU) at ($ (SrcA_mux) !0.5! (SrcB_mux) + (2,0) $) {\rotatebox{90}{ALU}};

    %ALU result register
    \node[register, align=center] (ALURes) at ($ (ALU) + (2,0) $) {ALU \\ Result};
    
    %ALU result mux
    \node[two input mux] (ALURes_mux) at ($ (ALURes) + (2,.225) $) {};
    \node[right, font=\tiny] at (ALURes_mux.lpin 1) {0};
    \node[right, font=\tiny] at (ALURes_mux.lpin 2) {1};

    %Sign extend
    \node[draw, rectangle, align=center, font=\scriptsize, minimum height=.7cm] at ($ (regfile) - (0,5) $) {Sign Extend};

    %<<2
    \node[trapezium,draw,trapezium left angle=120,trapezium right angle=60,trapezium stretches=true,minimum height=.5cm, font=\scriptsize\ttfamily] (ll2) at ($ (SrcB_mux) - (1,2) $) {<<2}; 

    %Control Unit
    \node[dipchip, num pins=14, hide numbers, no topmark, align=center, external pins width=0, circuitikz/multipoles/dipchip/pin spacing = 0.5] (CU) at (12.5,5) {\rotatebox{90}{\scriptsize \textbf{Control Unit}}};
    \draw (CU.bpin 2) ++(0,0.2) -- ++(0.2,-0.2) node[right, font=\tiny] {CLK} -- ++(-0.2,-0.2);
    \node[right, font=\tiny] at (CU.bpin 3) {IorD};
    \node[right, font=\tiny] at (CU.bpin 4) {MemWrite};
    \node[right, font=\tiny] at (CU.bpin 5) {IRWrite};
    \node[right, font=\tiny] at (CU.bpin 6) {Op};
    \node[right, font=\tiny] at (CU.bpin 7) {Funct};
    \node[left, font=\tiny] at (CU.bpin 8) {RegWrite};
    \node[left, font=\tiny] at (CU.bpin 9) {ALUSrcA};
    \node[left, font=\tiny] at (CU.bpin 10) {ALUSrcB};
    \node[left, font=\tiny] at (CU.bpin 11) {ALUControl};
    \node[left, font=\tiny] at (CU.bpin 12) {PCSrc};
    \node[left, font=\tiny] at (CU.bpin 13) {Branch};
    \node[left, font=\tiny] at (CU.bpin 14) {PCWrite};

    %-------------------------

\end{tikzpicture}

\end{document}