% Lezione del 28/04/2021

\documentclass[../main.tex]{subfiles}

\begin{document}
    \chapter{Architettura MIPS}
    Approfondimento del MIPS con visione della sua architettura.
    \section{Introduzione}
    Qui di seguito è rappresentato un esempio semplificato
    dell'architettura MIPS.

    \begin{figure}[h!]
        \centering

        \vspace*{1cm}
        \includegraphics[width=\linewidth]{image/MIPS_architecture}
        \newline
        \caption{Architettura MIPS semplificata}
        \vspace*{1cm}
    \end{figure}

    \noindent
    Il disegno si divide in due parti:
    \begin{itemize}
        \item la parte nera è l'unità di elaborazione (Data Path)
        \item la parte blu è l'unità di controllo (Control Unit)
    \end{itemize}

    \newpage

    \subsection{Memoria}
    La memoria non fa parte del processore (è esterna). \\[3mm]
    La memoria ha:
    \begin{itemize}
        \item un Address Bus, \texttt{Adr}
        \item un Data Bus "diviso" in due parti:
        \begin{itemize}
            \item una parte che mi da l'informazione
            nel caso in cui sto facendo una lettura, \texttt{RD}
            \item una parte che contiene il dato che devo scrivere \texttt{WD}
        \end{itemize}
    \end{itemize}
    
    \begin{figure}[h!]
        \centering

        \vspace*{5mm}
        \includegraphics[width=\linewidth]{image/MIPS_architecture}
        \newline
        \caption{Architettura MIPS - Memoria}
        \vspace*{5mm}
    \end{figure}

    Ad ogni colpo di clock, la memoria esegue una determinata operazione
    (lettura o scrittura) in base al valore del segnale di controllo
    "Write Enable" (\texttt{WE}):
    \begin{itemize}
        \item se vale 0 leggo in \texttt{RD} quello che c'è all'indirizzo di \texttt{A}
        \item se vale 1 scrivo il contenuto di \texttt{WD} all'indirizzo specificato in \texttt{A}
    \end{itemize}
    \vspace*{1mm}
    \underline{\textbf{Nota bene}: La memoria non ha
    un segnale di abilitazione.}

    \newpage

    \subsection{Il pacchetto di registri}
    In totale esistono 32 registri nel MIPS (codificati da 5 bit).

    \begin{figure}[h!]
        \centering

        \vspace*{5mm}
        \includegraphics[width=\linewidth]{image/MIPS_architecture}
        \newline
        \caption{Architettura MIPS - Pacchetto dei registri}
        \vspace*{5mm}
    \end{figure}

    Il set di registri è una memoria di tipo "Dual port" ossia mi
    permette di leggere ad ogni colpo di clock 2 word e quindi
    specifico 2 indirizzi. \\[1mm]
    Le parole \texttt{A1} e \texttt{A2} portano portano un'informazione
    di qual'è il registro che andrò a leggere
    in \texttt{RD1} e in \texttt{RD2}. \\
    Il set di registri è pilotato da un segnale di controllo
    \texttt{WE3} che:
    \begin{itemize}
        \item in lettura va a leggere il contenuto del registro
        specificato:
        \begin{itemize}
            \item o da \texttt{A1} e lo legge in \texttt{RD1}
            \item o da \texttt{A2} e lo legge in \texttt{RD2}
        \end{itemize}
        \item in scrittura va a prendere l'indirizzo in \texttt{A3}
        e il dato in \texttt{WD3}
    \end{itemize}

    \subsection{I registri}

    \begin{figure}[h!]
        \centering

        \vspace*{5mm}
        \includegraphics[width=\linewidth]{image/MIPS_architecture}
        \newline
        \caption{Architettura MIPS - Pacchetto dei registri}
        \vspace*{5mm}
    \end{figure}

    \noindent
    \underline{\textbf{Nota bene}: Per tutti i registri laddove non c'è un
    segnale di controllo vuol dire che a ogni colpo di clock} \\
    \underline{caricano ciò che c'è al loro ingresso, se non sono
    disabilitati (il PC e l'IR sono gli unici che possono essere} \\
    \underline{disabilitati).}

    \newpage
\end{document}