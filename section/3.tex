% Lezione del 29/03/2021

\documentclass[../main.tex]{subfiles}

\begin{document}

\section{Codifica delle costanti}
Poiché l'istruzione che contiene un immediato lo codifica
su 16 bit, come si fa a lavorare con 32 bit?
\\[2mm]
\textbf{Esempio} \\
Voglio caricare nel registro \texttt{\$s0} il contenuto di una costante
(è un numero). È su 16 o 32 bit?
\\[1mm]
\begin{itemize}
    \vspace*{-3mm}
    \item se è su 16 bit questa operazione si fa con una normale istruzione
    di \texttt{addi}:
    \\[1mm]
    \begin{tabular}{ p{9cm} p{7cm} }
        \textbf{Codice C} & \textbf{Codice MIPS assembly} \\
        \texttt{// int è una word a 32 bit (con segno)} & \texttt{\# \$s0 = a} \\
        \texttt{a = 0x4f3c;} & \texttt{addi \$s0, \$0, 0x4f3c} \\
    \end{tabular}
    \\[1mm]
    \underline{\texttt{Estensione del segno, 16 $\rightarrow$ 32 bit}}
    \vspace*{3mm}
    \item se è su 32 bit si usano la \texttt{lui} e la \texttt{ori}:
    \begin{enumerate}
        \item si "spezza" la \texttt{word} da 32 bit in 2 parti (alta
        e bassa) da 16 bit
        \item si prende la parte alta e si esegue la \texttt{lui}
        in un registro
        \item si prende la parte bassa e si esegue la \texttt{ori}
        tra il registro dove ho salvato la parte alta ed il valore della
        parte bassa salvando il risultato nello stesso registro di partenza
    \end{enumerate}
    \vspace*{1mm}
    \begin{tabular}{ p{9cm} p{7cm} }
        \textbf{Codice C} & \textbf{Codice MIPS assembly} \\
        \texttt{// int è una word a 32 bit (con segno)} & \texttt{\# \$s0 = a} \\
        \texttt{a = 0xFEDC8765;} & \texttt{lui \$s0, 0xFEDC} \\
        & \texttt{ori \$s0, \$s0, 0x8765} \\
    \end{tabular}
\end{itemize}

\vspace*{1mm}

\noindent
\underline{\textbf{Nota bene}: queste istruzioni sono di tipo I in quanto
sono presenti dei valori immediati (la \texttt{lui} ha un} \\
\underline{registro sorgente non utilizzato, posto a 0).}

\section{Moltiplicazione e divisione}
Le istruzioni di moltiplicazione e divisione utilizzano 2
registri particolari (aggiuntivi, fuori dalla categoria dei
32 registri): \texttt{lo}, \texttt{hi}. \\
Sono dei registri interni su cui noi non abbiamo modo di agire a meno
che si utilizzino determinate istruzioni.

\begin{table}[h!]
    \hspace*{1cm}
    \begin{minipage}{\linewidth}
        \subsection*{Moltiplicazione}
        La moltiplicazione di 2 numeri a 32 bit può creare un problema
        di overflow ovvero potrebbe non essere possibile
        rappresentare il dato su 32 bit. \\
        Per tutelarsi, MIPS restituisce un risultato a 64 bit. \\
        \hspace*{4mm} \texttt{mult \$s0, \$s1 \hspace*{0cm} \# risultato contenuto
        in \{hi, lo\}, registri a 32 bit} \\
        La parte alta del risultato viene salvata in \texttt{hi} e la parte bassa
        in \texttt{lo}.
    \end{minipage}

    \vspace*{5mm}

    \hspace*{1cm}
    \begin{minipage}{\linewidth}
        \subsection*{Divisione}
        La divisione di due numeri a 32 bit memorizza:
        \begin{itemize}
            \item il quoziente nel registro \texttt{lo}
            \item il resto nel registro \texttt{hi}
        \end{itemize}
    \end{minipage}
\end{table}

\noindent
Per ottenere i valori contenuti nei registri esistono 2
istruzioni specifiche:
\begin{itemize}
    \item la ''\textit{move from lo}" (\texttt{mflo}) che copia il valore
    del registro \texttt{lo} nel registro passato tramite parametro
    \item la ''\textit{move from hi}" (\texttt{mfhi}) che copia il valore
    del registro \texttt{hi} nel registro passato tramite parametro
\end{itemize}

\vspace*{2mm}

\noindent
\underline{\textbf{Nota bene}: queste istruzioni sono di tipo R in quanto
non sono presenti dei valori immediati.}

\subsection*{Esempio}
\texttt{mult \$s0, \$s1 \# risultato -> parte alta in 'hi', parte bassa in 'lo'} \\
\texttt{div \hspace*{0cm} \$s0, \$s1 \# risultato -> quoziente in 'lo', resto in 'hi'}\\
\\
\texttt{mflo \$s2 \# copia il contenuto del registro 'lo' nel registro 's2'} \\
\texttt{mfhi \$s3 \# copia il contenuto del registro 'hi' nel registro 's3'}

\newpage

\chapter{Pseudoistruzioni}
Le pseudoistruzioni (anche chiamate macroistruzioni) sono istruzioni
che vengono codificate in una o più istruzioni macchina (a differenza
delle normali istruzioni MIPS). \\
\underline{Sono state create per rendere i programmi più leggibili.}

\section{Load immediate}
La ''\textit{load immediate}" (\texttt{li}) è una pseudoistruzione
(non è un'istruzione MIPS e quindi non si può identificare in una
tipologia di istruzioni).
\\[2mm]
\textbf{Esempio} \\
\texttt{li \$t0, 4} viene tradotta in \texttt{ori \$t0, \$0, 4}
\\[2mm]
\textbf{Esempio} \\
Se ho dei numeri da 64 bit, per esempio:

\begin{center}
    \setlength{\tabcolsep}{30pt}
    \begin{tabular}{ c | c }
        \texttt{li \$t0, 90000} & \texttt{li \$t0, -5} \\
    \end{tabular}
\end{center}

\noindent
vengono tradotti in un altro modo. \\
\texttt{li \$t0, 90000}, per esempio, viene tradotta in \\
\texttt{\hspace*{0cm} \hspace*{0cm} \hspace*{0cm} lui \textbf{\$at}, 1 \# carica la parte superiore (uguale a 65536 (2$^\text{16}$))} \\
\texttt{\hspace*{0cm} \hspace*{0cm} \hspace*{0cm} ori \$t0, \textbf{\$at}, 24464 \# carica la parte inferiore (per arrivare a 90000)}
\\[1mm]
\underline{\textbf{Nota bene}: il registro \texttt{\textbf{\$at}} viene usato solamente per le pseudoistruzioni.}

\section{Load address}
La ''\textit{load address}" (\texttt{la}) è una pseudoistruzione
(non è un'istruzione MIPS e quindi non si può identificare in una
tipologia di istruzioni).
\\[2mm]
\textbf{Esempio} \\
\texttt{la \$t0, label} viene tradotta in \\
\texttt{\hspace*{0cm} \hspace*{0cm} \hspace*{0cm} lui \textbf{\$at}, n \# load upper 16 bits of label} \\
\texttt{\hspace*{0cm} \hspace*{0cm} \hspace*{0cm} ori \$t0, \textbf{\$at}, m \# lower 16 bits of label}

\section{Altre pseudoistruzioni}
La moltiplicazione (\texttt{mul}) e la divisione sono delle macroistruzioni
in quanto si basano su più istruzioni.
\begin{table}[h!]
    % \setlength{\tabcolsep}{0pt}
    \begin{tabular}{ p{3.5cm} p{3.5cm} p{3.5cm} p{6mm} p{3.5cm} }
        \multicolumn{2}{ c }{\hspace*{-1cm} \textbf{Moltiplicazione}} & \multicolumn{3}{ c }{\hspace*{-1cm} \textbf{Divisione}} \\
        \multirow{4}{*}{\texttt{mul rd, rs, rt}} & & \multirow{4}{*}{\texttt{div rd, rs, rt}} & & \texttt{bne rt, \$0, ok} \\
        & \texttt{mult rs, rt} & & & \texttt{break \$0} \\
        & \texttt{mflo rd} & & \texttt{ok:} & \texttt{div rs, rt} \\
        & & & & \texttt{mflo rd}
    \end{tabular}
\end{table}

\noindent
\underline{\textbf{Nota bene}: la \texttt{mul} si utilizza solamente se
siamo sicuri che il risultato sia su 32 bit.}

\newpage

\section*{Esempio}
Il seguente esercizio serve per calcolare il volume e l'area di un
parallelepipedo. Le formule sono le seguenti:

\texttt{volume = aSide $\cdot$ bSide $\cdot$ cSide}

\texttt{surfaceArea = 2 $\cdot$ (aSide $\cdot$ bSide + aSide $\cdot$ cSide + bSide $\cdot$ cSide)}

\vspace*{2mm}

\noindent
\textbf{Codice} \\
\lstinputlisting{example/3.0_parallelepipedo.asm}

\newpage

\chapter{Costrutti di alto livello}
I comuni linguaggi di alto livello hanno istruzioni che mi permettono
di fare determinate operazioni standard, tra cui:
\vspace*{-2mm}
\begin{table}[h!]
    \centering

    \setlength{\tabcolsep}{18pt}
    \begin{tabular}{ c c c c c }
        \texttt{if / else} & \texttt{cicli} & \texttt{vettori / matrici} & \texttt{funzioni} \\
    \end{tabular}
\end{table}
\vspace*{-2mm}

\noindent
Per creare questi costrutti in MIPS è necessario l'utilizzo
delle istruzioni di salto.

\section{Controllo del flusso}
Le seguenti istruzioni servono per fare i salti e permettono
di definire un controllo del flusso all'interno del codice.
I salti possono essere di diverso tipo:
\begin{itemize}
    \item condizionali
    \item incondizionali
\end{itemize}

\vspace{2mm}

\noindent
\underline{Queste istruzioni utilizzano una label, ovvero un identificatore
che corrisponde ad un indirizzo di codice} \\
\underline{(posizione di un'istruzione).}
\\[2mm]
Se l'istruzione di salto viene eseguita
\begin{itemize}
    \item allora vado direttamente al pezzo di codice contrassegnato
    dalla label e lo eseguo (saltando l'esecuzione del codice precedente)
    \item altrimenti eseguo tutto il codice e, arrivato alla label,
    eseguo anche il codice contenuto al suo interno
\end{itemize}

\subsection{Salto condizionale}
È un tipo di salto che valuta una condizione.
\begin{itemize}
    \item se due operandi sono uguali, si usa la \texttt{beq}
    \item se due operandi sono diversi, si usa la \texttt{bne}
\end{itemize}

\vspace{2mm}

\noindent
Queste istruzioni sono di tipo I (con 16 bit di label nel campo
\texttt{imm}).
\\[2mm]
\underline{\textbf{Nota bene}: la label è un immediato (indirizzo
$\rightarrow$ valore numerico) da 16 bit (l'indirizzo di memoria è da} \\
\underline{26 bit ma in seguito si vedrà come convertirlo a 16...).}

\vspace*{2mm}

\subsubsection{Istruzione \texttt{beq}}
L'istruzione \texttt{beq} fa un confronto tra i due registri e, se il loro contenuto
è uguale, allora salta nel pezzo di codice identificato dalla label.
\\[2mm]
\textbf{Codice MIPS assembly} \\
\texttt{addi \$s0, \$0, 4 \hspace*{0cm} \hspace*{0cm} \hspace*{0cm} \hspace*{0cm} \hspace*{0cm} \hspace*{0cm} \# \$s0 = 0 + 4 = 4} \\
\texttt{addi \$s1, \$0, 1 \hspace*{0cm} \hspace*{0cm} \hspace*{0cm} \hspace*{0cm} \hspace*{0cm} \hspace*{0cm} \# \$s1 = 0 + 1 = 1} \\
\texttt{sll \$s1, \$s1, 2 \hspace*{0cm} \hspace*{0cm} \hspace*{0cm} \hspace*{0cm} \hspace*{0cm} \hspace*{0cm} \# \$s1 = 1 $<<$ 2 = 4} \\
\texttt{beq \$s0, \$s1, target \hspace*{0cm} \# condizione verificata (=)} \\
\texttt{addi \$s1, \$s1, 1 \hspace*{0cm} \hspace*{0cm} \hspace*{0cm} \hspace*{0cm} \hspace*{0cm} \# non eseguita} \\
\texttt{sub \$s1, \$s1, \$s0 \hspace*{0cm} \hspace*{0cm} \hspace*{0cm} \hspace*{0cm} \# non eseguita} \\
\\
\texttt{target: \hspace*{0cm} \hspace*{0cm} \hspace*{0cm} \hspace*{0cm} \hspace*{0cm} \hspace*{0cm} \hspace*{0cm} \hspace*{0cm} \hspace*{0cm} \hspace*{0cm} \hspace*{0cm} \hspace*{0cm} \hspace*{0cm} \# label} \\
\texttt{\hspace*{0cm} \hspace*{0cm} add \$s1, \$s1, \$s0 \hspace*{0cm} \hspace*{0cm} \# \$s1 = 4 + 4 = 8}

\vspace*{2mm}

\subsubsection{Istruzione \texttt{bne}}
L'istruzione \texttt{bne} fa un confronto tra i due registri e, se il loro contenuto
è diverso, allora salta nel pezzo di codice identificato dalla label.
\\[2mm]
\textbf{Codice MIPS assembly} \\
\texttt{addi \$s0, \$0, 4 \hspace*{0cm} \hspace*{0cm} \hspace*{0cm} \hspace*{0cm} \hspace*{0cm} \hspace*{0cm} \# \$s0 = 0 + 4 = 4} \\
\texttt{addi \$s1, \$0, 1 \hspace*{0cm} \hspace*{0cm} \hspace*{0cm} \hspace*{0cm} \hspace*{0cm} \hspace*{0cm} \# \$s1 = 0 + 1 = 1} \\
\texttt{sll \$s1, \$s1, 2 \hspace*{0cm} \hspace*{0cm} \hspace*{0cm} \hspace*{0cm} \hspace*{0cm} \hspace*{0cm} \# \$s1 = 1 $<<$ 2 = 4} \\
\texttt{bne \$s0, \$s1, target \hspace*{0cm} \# condizione non verificata ($\neq$)} \\
\texttt{addi \$s1, \$s1, 1 \hspace*{0cm} \hspace*{0cm} \hspace*{0cm} \hspace*{0cm} \hspace*{0cm} \# \$s1 = 4 + 1 = 5} \\
\texttt{sub \$s1, \$s1, \$s0 \hspace*{0cm} \hspace*{0cm} \hspace*{0cm} \hspace*{0cm} \# \$s1 = 5 $–$ 4 = 1} \\
\\
\texttt{target: \hspace*{0cm} \hspace*{0cm} \hspace*{0cm} \hspace*{0cm} \hspace*{0cm} \hspace*{0cm} \hspace*{0cm} \hspace*{0cm} \hspace*{0cm} \hspace*{0cm} \hspace*{0cm} \hspace*{0cm} \hspace*{0cm} \# label} \\
\texttt{\hspace*{0cm} \hspace*{0cm} add \$s1, \$s1, \$s0 \hspace*{0cm} \hspace*{0cm} \# \$s1 = 1 + 4 = 5} \\

\newpage

\subsection{Salto incondizionale}
È un tipo di salto che viene effettuato senza nessuna condizione predefinita.
\begin{itemize}
    \item in questa categoria ci sono le istruzioni
    \texttt{j}, \texttt{jr} e \texttt{jal}
\end{itemize}

\vspace*{-4mm}

\subsubsection{Istruzione \texttt{j}}
L'istruzione \texttt{j} serve per saltare ad un pezzo di codice
contrassegnato da una label.
\\[2mm]
\textbf{Codice MIPS assembly} \\
\texttt{addi \$s0, \$0, 4 \hspace*{0cm} \hspace*{0cm} \hspace*{0cm} \hspace*{0cm} \hspace*{0cm} \hspace*{0cm} \# \$s0 = 0 + 4 = 4} \\
\texttt{addi \$s1, \$0, 1 \hspace*{0cm} \hspace*{0cm} \hspace*{0cm} \hspace*{0cm} \hspace*{0cm} \hspace*{0cm} \# \$s1 = 1} \\
\texttt{j target \hspace*{0cm} \hspace*{0cm} \hspace*{0cm} \hspace*{0cm} \hspace*{0cm} \hspace*{0cm} \hspace*{0cm} \hspace*{0cm} \hspace*{0cm} \hspace*{0cm} \hspace*{0cm} \hspace*{0cm} \hspace*{0cm} \# salta a target} \\
\texttt{sra \$s1, \$s1, 2 \hspace*{0cm} \hspace*{0cm} \hspace*{0cm} \hspace*{0cm} \hspace*{0cm} \hspace*{0cm} \# non eseguita} \\
\texttt{addi \$s1, \$s1, 1 \hspace*{0cm} \hspace*{0cm} \hspace*{0cm} \hspace*{0cm} \hspace*{0cm} \# non eseguita} \\
\texttt{sub \$s1, \$s1, \$s0 \hspace*{0cm} \hspace*{0cm} \hspace*{0cm} \hspace*{0cm} \# non eseguita} \\
\\
\texttt{target:} \\
\texttt{\hspace*{0cm} \hspace*{0cm} add \$s1, \$s1, \$s0 \hspace*{0cm} \hspace*{0cm} \# \$s1 = 1 + 4 = 5} \\

\begin{table}[h!]
    \centering

    \caption*{\texttt{j} è un'istruzione di \textbf{tipo J}}
    \setlength{\tabcolsep}{0pt}
    \begin{tabular}{ c c | c | c }
        \vspace*{-4.2mm} & \multicolumn{1}{ p{(\linewidth / 48) * 6} }{} & \multicolumn{1}{ p{((\linewidth / 48) * 26)} }{} \\
        \cline{2-3}
        \multicolumn{1}{ c | }{} & \texttt{\textbf{op}} & \texttt{\textbf{addr}} & \\
        \cline{2-3}
        \rule{0pt}{.8\normalbaselineskip} & \multicolumn{1}{ c }{6 bits} & \multicolumn{1}{ c }{26 bits} & \\
    \end{tabular}
\end{table}

\vspace*{-10mm}

\subsubsection{Istruzione \texttt{jr}}
L'istruzione \texttt{jr} serve per saltare ad un'istruzione specifica
memorizzando il suo indirizzo (posizione) \\ in un registro.
\\[2mm]
\textbf{Codice MIPS assembly} \\
\begin{table}[h!]
    \vspace*{-5mm}
    \hspace*{-.87cm}
    \setlength{\tabcolsep}{18pt}
    \begin{tabular}{ p{1.75cm} l }
        \textbf{\texttt{0x00002000}} & \texttt{addi \$s0, \$0, 0x2010} \\
        \textbf{\texttt{0x00002004}} & \texttt{jr \$s0} \\
        \textbf{\texttt{0x00002008}} & \texttt{addi \$s1, \$0, 1} \\
        \textbf{\texttt{0x0000200C}} & \texttt{sra \$s1, \$s1, 2} \\
        \textbf{\texttt{0x00002010}} & \texttt{lw \$s3, 44(\$s1)} \\
    \end{tabular}
\end{table}

\begin{table}[h!]
    \centering

    \caption*{\texttt{jr} è un'istruzione di \textbf{tipo R}}
    \setlength{\tabcolsep}{0pt}
    \begin{tabular}{ c c | c | c | c | c | c | c }
        \vspace*{-4.2mm} & \multicolumn{1}{ p{(\linewidth / 48) * 6} }{} & \multicolumn{1}{ p{(\linewidth / 48) * 5} }{} & \multicolumn{1}{ p{(\linewidth / 48) * 5} }{} & \multicolumn{1}{ p{(\linewidth / 48) * 5} }{} & \multicolumn{1}{ p{(\linewidth / 48) * 5} }{} & \multicolumn{1}{ p{(\linewidth / 48) * 6} }{} \\
        \cline{2-7}
        \multicolumn{1}{ c | }{} & \texttt{\textbf{0}} & \texttt{\textbf{rs}} & \texttt{0} & \texttt{0} & \texttt{0} & \texttt{\textbf{9}} & \\
        \cline{2-7}
        \rule{0pt}{.8\normalbaselineskip} & \multicolumn{1}{ c }{6 bits} & \multicolumn{1}{ c }{5 bits} & \multicolumn{1}{ c }{5 bits} & \multicolumn{1}{ c }{5 bits} & \multicolumn{1}{ c }{5 bits} & \multicolumn{1}{ c }{6 bits} & \\
    \end{tabular}
\end{table}

\noindent
\underline{Con questa istruzione, la CPU aggiorna il Program Counter con
l'indirizzo contenuto nel registro che ho} \\
\underline{passato come parametro alla \texttt{jr}.}

\section{Implementazione costrutti}
Implementazione di alcuni costrutti di alto livello in MIPS.

\subsection{if then}
\vspace*{-2mm}
Se la condizione è verificata
\begin{itemize}
    \item allora eseguo la somma e la differenza
    \item altrimenti eseguo solamente la differenza
\end{itemize}

\vspace*{2mm}

\noindent
\begin{tabular}{ p{7cm} p{7cm} }
    \textbf{Codice C} & \textbf{Codice MIPS assembly} \\
    & \texttt{\# \$s0 = f, \$s1 = g, \$s2 = h} \\
	\texttt{if (i == j)} & \texttt{\# \$s3 = i, \$s4 = j} \\
    \texttt{ \hspace*{0cm} f = g + h;} & \texttt{ \hspace*{0cm} bne \$s3, \$s4, L1} \\
    & \texttt{ \hspace*{0cm} add \$s0, \$s1, \$s2} \\
    \texttt{f = f $–$ i;} \\
    & \texttt{L1: sub \$s0, \$s0, \$s3} \\
\end{tabular}

\vspace*{2.5mm}

\noindent
\underline{\textbf{Nota bene}: assembly testa i casi negativi (i != j)
al contrario dell'alto livello (i == j).}

\subsection{if then/else}
\vspace*{-2mm}
Per creare il ramo \texttt{else} devo utilizzare un'istruzione di salto
incondizionato.

\vspace*{2.5mm}

\noindent
\begin{tabular}{ p{7cm} p{7cm} }
    \textbf{Codice C} & \textbf{Codice MIPS assembly} \\
    & \texttt{\# \$s0 = f, \$s1 = g, \$s2 = h} \\
	\texttt{if (i == j)} & \texttt{\# \$s3 = i, \$s4 = j} \\
    \texttt{ \hspace*{0cm} \hspace*{0cm} \hspace*{0cm} f = g + h;} & \texttt{ \hspace*{0cm} \hspace*{0cm} \hspace*{0cm} \hspace*{0cm} \hspace*{0cm} bne \$s3, \$s4, L1} \\
    \texttt{else} & \texttt{ \hspace*{0cm} \hspace*{0cm} \hspace*{0cm} \hspace*{0cm} \hspace*{0cm} add \$s0, \$s1, \$s2} \\
    \texttt{ \hspace*{0cm} \hspace*{0cm} \hspace*{0cm} f = f $–$ i;} & \texttt{ \hspace*{0cm} \hspace*{0cm} \hspace*{0cm} \hspace*{0cm} \hspace*{0cm} \textbf{j done}} \\
    & \texttt{L1: \hspace*{0cm} sub \$s0, \$s0, \$s3} \\
    \\
    & \texttt{done:} \\
\end{tabular}

\vspace*{2.5mm}

\noindent
\begin{tabular}{ p{7cm} p{7cm} }
    \textbf{Codice C} & \textbf{Codice MIPS assembly} \\
    & \texttt{\# \$s0 = f} \\
	\texttt{if (i != j)} & \texttt{\# \$s3 = i, \$s4 = j} \\
    \texttt{ \hspace*{0cm} \hspace*{0cm} \hspace*{0cm} f++;} & \texttt{ \hspace*{0cm} \hspace*{0cm} \hspace*{0cm} \hspace*{0cm} \hspace*{0cm} \textbf{beq \$s3, \$s4, L1}} \\
    \texttt{else} & \texttt{ \hspace*{0cm} \hspace*{0cm} \hspace*{0cm} \hspace*{0cm} \hspace*{0cm} addi \$s0, \$s0, 1} \\
    \texttt{ \hspace*{0cm} \hspace*{0cm} \hspace*{0cm} f--;} & \texttt{ \hspace*{0cm} \hspace*{0cm} \hspace*{0cm} \hspace*{0cm} \hspace*{0cm} \textbf{j done}} \\
    & \texttt{L1: \hspace*{0cm} addi \$s0, \$s0, -1} \\
    \\
    & \texttt{done:} \\
\end{tabular}

\subsection{while}
\vspace*{-2mm}
Per implementare un \texttt{while} bisogna utilizzare un'istruzione
di salto condizionale seguita da una incondizionale.

\vspace*{2.5mm}

\begin{tabular}{ p{7cm} p{7cm} }
    \textbf{Codice C} & \textbf{Codice MIPS assembly} \\
    \texttt{// determina la potenza} & \texttt{\# \$s0 = pow, \$s1 = x} \\
    \texttt{// di x come $2^\text{x}$ = 128} \\
    \texttt{int pow = 1;} & \texttt{ \hspace*{0cm} \hspace*{0cm} \hspace*{0cm} \hspace*{0cm} \hspace*{0cm} \hspace*{0cm} \hspace*{0cm} addi \$s0, \$0, 1} \\
    \texttt{int x = 0;} & \texttt{ \hspace*{0cm} \hspace*{0cm} \hspace*{0cm} \hspace*{0cm} \hspace*{0cm} \hspace*{0cm} \hspace*{0cm} add \$s1, \$0, \$0} \\
    & \texttt{ \hspace*{0cm} \hspace*{0cm} \hspace*{0cm} \hspace*{0cm} \hspace*{0cm} \hspace*{0cm} \hspace*{0cm} addi \$t0, \$0, 128} \\
    \texttt{while (pow != 128) \{} & \texttt{while: beq \$s0, \$t0, done} \\
    \texttt{\hspace*{0cm} \hspace*{0cm} \hspace*{0cm} pow = pow * 2;} & \texttt{ \hspace*{0cm} \hspace*{0cm} \hspace*{0cm} \hspace*{0cm} \hspace*{0cm} \hspace*{0cm} \hspace*{0cm} sll \$s0, \$s0, 1} \\
    \texttt{\hspace*{0cm} \hspace*{0cm} \hspace*{0cm} x = x + 1;} & \texttt{ \hspace*{0cm} \hspace*{0cm} \hspace*{0cm} \hspace*{0cm} \hspace*{0cm} \hspace*{0cm} \hspace*{0cm} addi \$s1, \$s1, 1} \\
    \texttt{\}} & \texttt{ \hspace*{0cm} \hspace*{0cm} \hspace*{0cm} \hspace*{0cm} \hspace*{0cm} \hspace*{0cm} \hspace*{0cm} j while} \\
    \\
    & \texttt{done:} \\
\end{tabular}

\vspace*{2.5mm}

\noindent
\underline{\textbf{Nota bene}: Assembly controlla per il caso
opposto (\texttt{pow == 128}) rispetto al codice C (\texttt{pow != 128}).}

\subsection{for}
\vspace*{-2mm}
Per implementare un \texttt{for} bisogna utilizzare un'istruzione
di salto condizionale che lavori con un contatore ed un
salto incondizionale.

\vspace*{2.5mm}

\noindent
\begin{tabular}{ p{9cm} p{9cm} }
    \textbf{Codice C} & \textbf{Codice MIPS assembly} \\
    \texttt{// aggiunge i numeri da 0 a 9} & \texttt{\# \$s0 = i, \$s1 = sum} \\
    \texttt{int sum = 0;} & \texttt{ \hspace*{0cm} \hspace*{0cm} \hspace*{0cm} \hspace*{0cm} \hspace*{0cm} addi \$s0, \$0, 0} \\
    \texttt{int i;} & \texttt{ \hspace*{0cm} \hspace*{0cm} \hspace*{0cm} \hspace*{0cm} \hspace*{0cm} add \$s0, \$0, \$0} \\
    & \texttt{ \hspace*{0cm} \hspace*{0cm} \hspace*{0cm} \hspace*{0cm} \hspace*{0cm} addi \$t0, \$0, 10} \\
    \texttt{for (i = 0; i != 10; i = i + 1) \{} & \texttt{for: beq \$s0, \$t0, done} \\
    \texttt{ \hspace*{0cm} \hspace*{0cm} \hspace*{0cm} sum = sum + i;} & \texttt{ \hspace*{0cm} \hspace*{0cm} \hspace*{0cm} \hspace*{0cm} \hspace*{0cm} add \$s1, \$s1, \$s0} \\
    \texttt{\}} & \texttt{ \hspace*{0cm} \hspace*{0cm} \hspace*{0cm} \hspace*{0cm} \hspace*{0cm} addi \$s0, \$s0, 1} \\
    & \texttt{ \hspace*{0cm} \hspace*{0cm} \hspace*{0cm} \hspace*{0cm} \hspace*{0cm} j for} \\
    \\
    & \texttt{done:} \\
\end{tabular}

\newpage

\section{Confronto di grandezze}
L'istruzione \textit{''Set on Less Than"} (\texttt{slt}) fa il confronto
tra due registri secondo l'ordine in cui compaiono.

\vspace*{-2mm}

\begin{table}[h!]
    \begin{minipage}{.5\linewidth}
        \texttt{slt rd, rs, rt}
        \begin{itemize}
            \item \texttt{if (rs $<$ rt) \\
            \hspace*{0cm} \hspace*{0cm} rd = 1; \\
            else \\
            \hspace*{0cm} \hspace*{0cm} rd = 0;}
        \end{itemize}
    \end{minipage}
    \begin{minipage}{.5\linewidth}
        \texttt{slti rt, rs, constant}
        \begin{itemize}
            \item \texttt{if (rs $<$ constant) \\
            \hspace*{0cm} \hspace*{0cm}  rt = 1; \\
            else \\
            \hspace*{0cm} \hspace*{0cm} rt = 0;}
        \end{itemize}
    \end{minipage}
\end{table}

\noindent
Usando in coppia la \texttt{slt} e la \texttt{bne} posso implementare
un salto condizionale con una condizione di minore tra due registri.
\\[2mm]
\texttt{slt \$t0, \$s1, \$s2 \# if (\$s1 $<$ \$s2)} \\
\texttt{bne \$t0, \$0, L \hspace*{0cm} \hspace*{0cm} \hspace*{0cm} \# \hspace*{0cm} \hspace*{0cm} branch to L}
\\[2mm]
Inoltre, è possibile lavorare con numeri:
\begin{itemize}
    \item con segno utilizzando le istruzioni \texttt{slt} e \texttt{slti}
    \item senza segno utilizzando le istruzioni \texttt{sltu} e \texttt{sltui}
\end{itemize}

\vspace*{5mm}

\noindent
\textbf{Esempio} \\
\texttt{\$s0 = 1111 1111 1111 1111 1111 1111 1111 1111} \\
\texttt{\$s1 = 0000 0000 0000 0000 0000 0000 0000 0001}
\begin{table}[h!]
    \begin{minipage}{.5\linewidth}
        \texttt{slt \$t0, \$s0, \$s1 \# signed} \\
        \texttt{\hspace*{0cm} \hspace*{0cm} $-$1 $<$ +1 $\Rightarrow$ \$t0 = 1} \\
    \end{minipage}
    \begin{minipage}{.5\linewidth}
        \texttt{sltu \$t0, \$s0, \$s1 \# unsigned} \\
\texttt{\hspace*{0cm} \hspace*{0cm} +4,294,967,295 $>$ +1 $\Rightarrow$ \$t0 = 0} \\
    \end{minipage}
\end{table}

\noindent
\textbf{Esempio} \\
Ciclo con la condizione di minore.
\\[3mm]
\begin{tabular}{ p{9cm} p{9cm} }
    \textbf{Codice C} & \textbf{Codice MIPS assembly} \\
    & \texttt{\# \$s0 = i, \$s1 = sum} \\
    \texttt{// incrementa con le potenze del 2 da 1} & \texttt{ \hspace{0cm} \hspace{0cm} \hspace{0cm} \hspace{0cm} \hspace{0cm} \hspace{0cm} addi \$s1, \$0, 0} \\
    \texttt{// a 100} & \texttt{ \hspace{0cm} \hspace{0cm} \hspace{0cm} \hspace{0cm} \hspace{0cm} \hspace{0cm} addi \$s0, \$0, 1} \\
    \texttt{int sum = 0;} & \texttt{ \hspace{0cm} \hspace{0cm} \hspace{0cm} \hspace{0cm} \hspace{0cm} \hspace{0cm}  addi \$t0, \$0, 101} \\
    \texttt{int i;} & \texttt{loop: slt \$t1, \$s0, \$t0} \\
    & \texttt{ \hspace{0cm} \hspace{0cm} \hspace{0cm} \hspace{0cm} \hspace{0cm} \hspace{0cm} beq \$t1, \$0, done} \\
    \texttt{for (i = 1; i $<$ 101; i = i * 2) \{} & \texttt{ \hspace{0cm} \hspace{0cm} \hspace{0cm} \hspace{0cm} \hspace{0cm} \hspace{0cm} add \$s1, \$s1, \$s0} \\
    \texttt{ \hspace{0cm} sum = sum + i;} & \texttt{ \hspace{0cm} \hspace{0cm} \hspace{0cm} \hspace{0cm} \hspace{0cm} \hspace{0cm} sll \$s0, \$s0, 1} \\
    \texttt{\}} & \texttt{ \hspace{0cm} \hspace{0cm} \hspace{0cm} \hspace{0cm} \hspace{0cm} \hspace{0cm} j loop} \\
    & \texttt{done:} \\
\end{tabular}

\noindent
\underline{\textbf{Nota bene}: \texttt{\$t1 = 1 se i $<$ 101}}

\section{Confronti con pseudoistruzioni}
Per semplificare la scrittura del codice, sono state implementate alcune
pseudoistruzioni per effettuare i confronti tra le grandezze. \\
Di seguito, è presente una lista delle pseudoistruzioni disponibili.

\begin{table}[h!]
    \begin{minipage}{.5\linewidth}
        \begin{tabular}{ l l }
            \multicolumn{2}{c}{\textbf{Istruzioni - Valori con segno}} \\
            \texttt{bge \$t0, \$s0, L1} & \texttt{slt \$at, \$t0, \$s0} \\
            & \texttt{beq \$at, \$0, L1} \\
            \texttt{blt \$t0, \$s0, L1} & \texttt{slt \$at, \$t0, \$s0} \\
            & \texttt{bne \$at, \$0, L1} \\
            \texttt{bgt \$t0, \$s0, L1} & \texttt{slt \$at, \$s0, \$t0} \\
            & \texttt{bne \$at, \$0, L1} \\
            \texttt{ble \$t0, \$s0, L1} & \texttt{slt \$at, \$s0, \$t0} \\
            & \texttt{beq \$at, \$0, L1} \\
        \end{tabular}
    \end{minipage}
    \begin{minipage}{.5\linewidth}
        \begin{tabular}{ l l }
            \multicolumn{2}{c}{\textbf{Istruzioni - Valori senza segno}} \\
            \texttt{bgeu \$t0, \$s0, L1} & \texttt{sltu \$at, \$t0, \$s0} \\
            & \texttt{beq \$at, \$0, L1} \\
            \texttt{bltu \$t0, \$s0, L1} & \texttt{sltu \$at, \$t0, \$s0} \\
            & \texttt{bne \$at, \$0, L1} \\
            \texttt{bgtu \$t0, \$s0, L1} & \texttt{sltu \$at, \$s0, \$t0} \\
            & \texttt{bne \$at, \$0, L1} \\
            \texttt{bleu \$t0, \$s0, L1} & \texttt{sltu \$at, \$s0, \$t0} \\
            & \texttt{beq \$at, \$0, L1} \\
        \end{tabular}
    \end{minipage}
\end{table}

\noindent
\textbf{Esempio} \\
L'istruzione \textit{''branch if less than"} (\texttt{blt})
confronta 2 registri, trattandoli come valori interi con segno
ed effettua un salto se il primo registro è minore del secondo. \\
Quindi, la pseudoistruzione \\
\texttt{\hspace*{5mm}blt \$8, \$9, label} \\
si traduce in \\
\texttt{\hspace*{5mm}slt \$1, \$8, \$9} \\
\texttt{\hspace*{5mm}bne \$1, \$0, label}

\newpage

\end{document}